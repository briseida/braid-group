\documentclass[xcolor=dvipsnames]{beamer}
\setbeamercovered{transparent}
\usepackage{beamerthemedefault}
\usepackage{polyglossia}
\setmainlanguage{spanish}
%\usepackage{beamerthemeIlmenau}
%\usepackage{color}  % for \textcolor{red}{blah}
%\usepackage[dvipsnames]{xcolor}
\usepackage{xcolor}
\usepackage{auto-pst-pdf}
\usecolortheme{spruce}
\usepackage{graphicx}
\usepackage{verbatim} % for \begin{comment}
\usepackage{amsmath,amscd,amsthm}
\usepackage{amssymb,latexsym}
\usepackage{amscd,amsfonts}
\usepackage[vcentermath]{youngtab}
\usepackage{mathrsfs,dsfont}
%%%\usepackage{pstricks}
\usepackage[usenames,dvipsnames]{pstricks}
\usepackage{stackengine}
\usepackage{multirow}
\usepackage{tikz}
\usepackage{pgfplots}
\pgfplotsset{compat=1.13}
\usepackage{braids}


\title{Braid Groups}
%\subtitle{\textbf{Subtitle}}
\author{Ernesto V\'azquez \\ Briseida Trejo}
\institute{UNAM}

\begin{document}

\maketitle
\frame{
\uncover<1->{
  The \alert{n-strand braid groups} $B_n$ has the presentation:
\begin{equation*}
B_n=\langle\sigma_1,\dots,\sigma_{n-1}\mid \sigma_i\sigma_j=\sigma_j\sigma_i \mbox{ for } |i-j|\geq 2,\sigma_i\sigma_{i+1}\sigma_i=\sigma_{i+1}\sigma_i\sigma_{i+1}\rangle.
\end{equation*} 
}
} 

\frame{
\frametitle{Pure braids}
\uncover<1->{
Two words $\beta_1$ and $\beta_2$ represent the same braid $\Leftrightarrow$ $\beta_1\beta_2^{-1}=1$.
}

\uncover<2->{
\begin{block}{Definition}
	Dibujo de pure braid
\end{block}
}

\uncover<3->{
\begin{block}{Theorem}
The pure braid group $P_n$ is biorderable for all $n\geq 1$.
\end{block}
}
}

\frame{
\begin{tikzpicture}
\braid a_1 a_2;
\end{tikzpicture}
}

\frame{
\begin{tikzpicture}
\braid[color=red] | a_1 a_2;
\end{tikzpicture}
\begin{tikzpicture}
\braid a_1 | a_2;
\end{tikzpicture}
}

\frame{
\begin{tikzpicture}
\braid[
 style all floors={fill=yellow},
 style floors={1}{dashed,fill=yellow!50!green},
 floor command={%
 \fill (\floorsx,\floorsy) rectangle (\floorex,\floorey);
 \draw (\floorsx,\floorsy) -- (\floorex,\floorsy);
 },
 line width=2pt,
 style strands={1}{red},
 style strands={2}{blue},
 style strands={3}{green}
] (braid) at (2,0) | s_1-s_3-s_5 | s_2^{-1}-s_4| s_1-s_4 s_2^{-1} s_1-s_3 s_2^{-1}-s_4^{-1};
\fill[yellow] (2,0) circle (4pt);
\fill[purple] (braid) circle (4pt);
\node[at=(braid-3-s),pin=north west:strand 3] {};
\node[at=(braid-3-e),pin=south west:strand 3] {};
\node[at=(braid-rev-3-s),pin=north east:strand 3 (from bottom)] {};
\node[at=(braid-rev-3-e),pin=south east:strand 3 (from bottom)] {};
\end{tikzpicture}
}

\frame{
\frametitle{Mapping class groups}
\uncover<1->{
\begin{block}{Definition}
Let $\mathcal{S}$ an oriented compact surface, possibly with boundary, and $\mathcal{P}$ be a finite set of distinguished interior points of $\mathcal{S}$.\\
}

\uncover<2->{
The \alert{mapping clas group} $\mathcal{MCG(S,P)}$ of the surface $\mathcal{S}$ relative to $\mathcal{P}$ is the group of all isotopy classes of orientation-preserving homeomorphisms $\psi:\mathcal{S}\rightarrow\mathcal{S}$ satisfying $\psi_{|\partial S}=id$ and $\psi(\mathcal{P})=\mathcal{P}$.
\end{block}
}

\uncover<3->{
\begin{block}{Proposition}
	There is an isomorphism $B_n\cong \mathcal{MCG}(D_n)$, where $D_n$ is the disk $D^2$ with $n$ regularly spaced points.
\end{block}
}
%\textcolor{red}{Dibujo?}
}

\frame{
\frametitle{Order in $B_n$}
$\sigma$-positive

subword property

\begin{block}{Proposition}
	
\end{block}

}

%\frame{
%\uncover<1->{
%  \begin{block}{Definición}
%    Dado $S$ un conjunto finito de finito de puntos,
%    el \alert{casco convexo} de $S$, es el conjunto convexo m\'as peque\~no que contiene a $S$.
%  \end{block}    
%}
%\uncover<2->{
%  \begin{block}{Definición}
%    Una \alert{$n$-politopo convexo} es el casco convexo de un conjunto finito de puntos que est\'an en un espacio de dimensi\'on $n$.
%  \end{block} 
%}  
%\invisible<1,2>{
%  \alert{Ejemplos}
%  \begin{figure}[!htbp]\centering \psset{xunit=0.8}\psset{yunit=0.8}
%  \begin{minipage}{0.9\linewidth}
%  \centering
%    %\resizebox{0.6\textwidth}{!}{\input{decagono.tex}}
% 
%   %\hspace{0.3cm} 2-politopo convexo \qquad 2-politopo no convexo 
%   \end{minipage}
%}
%\end{figure}
%} 

%\frame{
%\begin{center}
%\frametitle{Torsiones derechas}
%\visible<1->{
%\begin{figure}[!htbp]\centering \psset{xunit=0.7}\psset{yunit=0.7}
%  \resizebox{0.25\textwidth}{!}{\input{Petrie-r0r1r2r3-coloreado.tex}}
%\end{figure}
%\vspace{0.4cm}
%}
% 
%\visible<2->{  
%\begin{minipage}{0.3\linewidth}
%   \centering
%  \begin{figure}[!htbp]\centering \psset{xunit=0.7}\psset{yunit=0.7}
%  \resizebox{0.8\textwidth}{!}{\input{torsion-derecha-1-1.tex}}
%  \end{figure}
%\end{minipage}
%\begin{minipage}{0.3\linewidth}
%      \centering
%  \begin{figure}[!htbp]\centering \psset{xunit=0.7}\psset{yunit=0.7}
%  \resizebox{0.8\textwidth}{!}{\input{torsion-derecha-2-1.tex}}
%  \end{figure}
%\end{minipage}
%\begin{minipage}{0.3\linewidth}
%      \centering
%  \begin{figure}[!htbp]\centering \psset{xunit=0.7}\psset{yunit=0.7}
%  \resizebox{0.79\textwidth}{!}{\input{torsion-derecha-3-1.tex}}
%  \end{figure}
%\end{minipage}
%}
%\end{center}
%}





\frame{
\begin{center}
\Huge Thanks.
\end{center}
}
%%%\resizebox{0.1\textwidth}{!}{\input{cubo.tex}}
\end{document}