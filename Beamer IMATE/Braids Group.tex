\documentclass{beamer}
\usepackage[utf8]{inputenc}
\usepackage[T1]{fontenc}
\usepackage[spanish,mexico]{babel}

\usepackage{amsthm}
\usepackage{verbatim}
\usepackage{paralist}
\usepackage{graphics}
\usepackage[all]{xy}
\usepackage{xcolor}
\usepackage{hyperref}

\usepackage{pgf,tikz}
\usepackage{mathrsfs}
\usetikzlibrary{arrows}

\usepackage{tikz}
\usepackage{pgfplots}
\usepackage{braids}

\definecolor{qqwuqq}{rgb}{0.,0.39215686274509803,0.}
\definecolor{uuuuuu}{rgb}{0.26666666666666666,0.26666666666666666,0.26666666666666666}
\definecolor{ffxfqq}{rgb}{1.,0.4980392156862745,0.}
\definecolor{ffqqqq}{rgb}{1.,0.,0.}
\definecolor{qqqqff}{rgb}{0.,0.,1.}
\definecolor{ffqqff}{rgb}{1.,0.,1.}
\definecolor{xfqqff}{rgb}{0.4980392156862745,0.,1.}
\definecolor{xdxdff}{rgb}{0.49019607843137253,0.49019607843137253,1.}
\definecolor{wwqqzz}{rgb}{0.4,0.,0.6}
\definecolor{qqffqq}{rgb}{0.,1.,0.}

\theoremstyle{definition}
\newtheorem{df}{Definición}
\newtheorem{teo}{Teorema}
\newtheorem{ej}{Ejemplo}
\newtheorem{prop}{Proposición}
\newtheorem{lema}{Lema}
\newtheorem{cor}{Corolario}
\newtheorem{obs}{Observación}

\newenvironment{pba}{\noindent\textbf{Prueba:}}{\begin{flushright} $\square$ \end{flushright}}
\newenvironment{dem}{\noindent\textbf{Demostración:}}{\begin{flushright} \rule{1ex}{1ex} \end{flushright}}

\usetheme{Imunam}


\newcommand{\N}{\mathbb N}
\newcommand{\Z}{\mathbb Z}
\newcommand{\R}{\mathbb R}
\newcommand{\C}{\mathbb C}
\newcommand{\h}{\mathbb H}
\newcommand{\D}{\Delta}
\newcommand{\s}{\mathbb S}



\title{Braid groups}
\author{Briseida Trejo \\ Ernesto Vázquez %\\ \texttt{evazquez@matem.unam.mx}
}
\date{Summer Graduate School MSRI-CMO: Geometric Group Theory \\ FECHA} %Fecha o evento en que se presentará la plática
\institute{Instituto de Matemáticas, UNAM}

\begin{document}

\begin{frame}
\titlepage %Necesario para generar la portada
\end{frame}


%%% PRELIMINARES
%\section{Preliminares}

%\begin{frame}{Título}
%puras chingonerías
%\end{frame}

\frame{
\frametitle{Braids}
\uncover<1->{
	\begin{block}{Definition}
		An \alert{$n$-braid} is a collection of $n$ disjoint strings.
	\end{block}
}
\uncover<2->{
	\begin{block}{Examples}
	\centering
	\begin{minipage}{0.2\linewidth}
  	\centering
	\begin{tikzpicture}[scale=0.5]
		\braid [number of strands=3,line width=2pt,style strands={1}{green},style strands={2}{yellow},style strands={3}{blue},style strands={4}{red}]
 			a_1 a_2 a_1;
	\end{tikzpicture}
	
	3-braid
	\end{minipage}\quad\quad
	\begin{minipage}{0.2\linewidth}
		\centering
	\begin{tikzpicture}[scale=0.5]
		\braid [number of strands=3,line width=2pt,style strands={1}{green},style strands={2}{yellow},style strands={3}{blue},style strands={4}{red},height=3cm]
		1;
	\end{tikzpicture}
	
	$id$
	\end{minipage}
	\end{block}
}
}

\frame{
\frametitle{Braid groups}
\uncover<1->{
  The \alert{n-strand braid group} $B_n$ has the presentation:
\begin{equation*}
B_n=\langle\sigma_1,\dots,\sigma_{n-1}\mid \sigma_i\sigma_j=\sigma_j\sigma_i \mbox{ for } |i-j|\geq 2,\sigma_i\sigma_{i+1}\sigma_i=\sigma_{i+1}\sigma_i\sigma_{i+1}\rangle.
\end{equation*}
}
\uncover<2->{
  \begin{block}{Generators of $B_4$}
  \centering 
  \begin{minipage}{0.25\linewidth}
  	\centering
	\begin{tikzpicture}[scale=0.5]
		\braid [number of strands=4,line width=2pt,style strands={1}{green},style strands={2}{yellow},style strands={3}{blue},style strands={4}{red}]
 			a_1;
	\end{tikzpicture}
	
	$\sigma_1$
	\end{minipage}
	\quad
%}
	\begin{minipage}{0.25\linewidth}
		\centering
	\begin{tikzpicture}[scale=0.5]
		\braid [number of strands=4,line width=2pt,style strands={1}{green},style strands={2}{yellow},style strands={3}{blue},style strands={4}{red}]
		a_2;
	\end{tikzpicture}
	
	$\sigma_2$
	\end{minipage}
	\quad
	\begin{minipage}{0.25\linewidth}
		\centering
	\begin{tikzpicture}[scale=0.5]
		\braid [number of strands=4,line width=2pt,style strands={1}{green},style strands={2}{yellow},style strands={3}{blue},style strands={4}{red}]
		a_3;
	\end{tikzpicture}
	
	$\sigma_3$
	\end{minipage}
	
	\vspace{0.7cm}
	\begin{minipage}{0.25\linewidth}
  	\centering
	\begin{tikzpicture}[scale=0.5]
		\braid [number of strands=4,line width=2pt,style strands={1}{green},style strands={2}{yellow},style strands={3}{blue},style strands={4}{red}]
 			a_1^{-1};
	\end{tikzpicture}
	
	$\sigma_1^{-1}$
	\end{minipage}
	\quad
%}
	\begin{minipage}{0.25\linewidth}
		\centering
	\begin{tikzpicture}[scale=0.5]
		\braid [number of strands=4,line width=2pt,style strands={1}{green},style strands={2}{yellow},style strands={3}{blue},style strands={4}{red}]
		a_2^{-1};
	\end{tikzpicture}
	
	$\sigma_2^{-1}$
	\end{minipage}
	\quad
	\begin{minipage}{0.25\linewidth}
		\centering
	\begin{tikzpicture}[scale=0.5]
		\braid [number of strands=4,line width=2pt,style strands={1}{green},style strands={2}{yellow},style strands={3}{blue},style strands={4}{red}]
		a_3^{-1};
	\end{tikzpicture}
	
	$\sigma_3^{-1}$
	\end{minipage}
	\end{block}
	}
}

\frame{
\frametitle{Relations}
%\uncover<1->{
  \centering 
  \begin{minipage}{0.2\linewidth}
  	\centering
	\begin{tikzpicture}[scale=0.5]
		\braid [number of strands=4,line width=2pt,style strands={1}{green},style strands={2}{yellow},style strands={3}{blue},style strands={4}{red}]
 			a_1 a_3;
	\end{tikzpicture}
	
	$\sigma_1\sigma_3$
	\end{minipage}
	=
%}
	\begin{minipage}{0.2\linewidth}
		\centering
	\begin{tikzpicture}[scale=0.5]
		\braid [number of strands=4,line width=2pt,style strands={1}{green},style strands={2}{yellow},style strands={3}{blue},style strands={4}{red}]
		a_3 a_1;
	\end{tikzpicture}
	
	$\sigma_3\sigma_1$
	\end{minipage}
	\quad\quad\quad
	\begin{minipage}{0.25\linewidth}
	$\sigma_i\sigma_j=\sigma_j\sigma_i$ \\ if $|i-j|\geq 2$
	\end{minipage}
	
	\vspace{0.7cm}
	  \begin{minipage}{0.2\linewidth}
  	\centering
	\begin{tikzpicture}[scale=0.5]
		\braid [number of strands=3,line width=2pt,style strands={1}{green},style strands={2}{yellow},style strands={3}{blue},style strands={4}{red}]
 			a_1 a_2 a_1;
	\end{tikzpicture}
	
	$\sigma_1\sigma_2\sigma_1$
	\end{minipage}
	=
%}
	\begin{minipage}{0.2\linewidth}
		\centering
	\begin{tikzpicture}[scale=0.5]
		\braid [number of strands=3,line width=2pt,style strands={1}{green},style strands={2}{yellow},style strands={3}{blue},style strands={4}{red}]
		a_2 a_1 a_2;
	\end{tikzpicture}
	
	$\sigma_2\sigma_1\sigma_2$
	\end{minipage}
	\quad
	\begin{minipage}{0.35\linewidth}
	$\sigma_i\sigma_{i+1}\sigma_i=\sigma_{i+1}\sigma_i\sigma_{i+1}$ 
	\end{minipage}
}

\frame{
\frametitle{Multiplication (Concatenation)}
%\uncover<1->{
  \centering 
  \begin{minipage}{0.2\linewidth}
  	\centering
	\begin{tikzpicture}[scale=0.5]
		\braid [number of strands=4,line width=2pt,style strands={1}{green},style strands={2}{yellow},style strands={3}{blue},style strands={4}{red}]
 			a_1;
	\end{tikzpicture}
	\end{minipage}
	*
%}
	\begin{minipage}{0.2\linewidth}
  	\centering
	\begin{tikzpicture}[scale=0.5]
		\braid [number of strands=4,line width=2pt,style strands={1}{green},style strands={2}{yellow},style strands={3}{blue},style strands={4}{red}]
 			a_1 a_2 a_1;
	\end{tikzpicture}
	\end{minipage}
	=
%}
	\begin{minipage}{0.2\linewidth}
		\centering
	\begin{tikzpicture}[scale=0.5]
		\braid [
		number of strands=4,line width=2pt,style strands={1}{green},style strands={2}{yellow},style strands={3}{blue},style strands={4}{red}]
		 a_1 a_1 a_2 a_1;
	\end{tikzpicture}
	\end{minipage}
}

\frame{
\frametitle{Multiplication (Concatenation)}
%\uncover<1->{
  \centering
  \fbox{ 
  \begin{minipage}{0.2\linewidth}
  	\centering
	\begin{tikzpicture}[scale=0.5]
		\braid [number of strands=4,line width=2pt,style strands={1}{green},style strands={2}{yellow},style strands={3}{blue},style strands={4}{red}]
 			a_1;
	\end{tikzpicture}
	\end{minipage}
	}
	*
%}
	\begin{minipage}{0.2\linewidth}
  	\centering
	\begin{tikzpicture}[scale=0.5]
		\braid [number of strands=4,line width=2pt,style strands={1}{green},style strands={2}{yellow},style strands={3}{blue},style strands={4}{red}]
 			a_1 a_2 a_1;
	\end{tikzpicture}
	\end{minipage}
	=
%}
	\begin{minipage}{0.2\linewidth}
		\centering
	\begin{tikzpicture}[scale=0.5]
		\braid [
		number of strands=4,line width=2pt,style strands={1}{green},style strands={2}{yellow},style strands={3}{blue},style strands={4}{red}]
		 |a_1 a_1 a_2 a_1;
	\end{tikzpicture}
	\end{minipage}
}

\frame{
\frametitle{The word problem}
\uncover<1->{
Two words $\beta_1,\beta_2\in B_n$ represent the same braid $\Leftrightarrow$ $\beta_1\beta_2^{-1}=1$.
}
  \centering 
\uncover<2->{
  \begin{minipage}{0.2\linewidth}
  	\centering
	\begin{tikzpicture}[scale=0.5]
		\braid [number of strands=4,line width=2pt,style strands={1}{green},style strands={2}{yellow},style strands={3}{blue},style strands={4}{red}]
 			a_1 a_3;
	\end{tikzpicture}
	
	$\sigma_1\sigma_3$
	\end{minipage}
	=
	\begin{minipage}{0.2\linewidth}
		\centering
	\begin{tikzpicture}[scale=0.5]
		\braid [number of strands=4,line width=2pt,style strands={1}{green},style strands={2}{yellow},style strands={3}{blue},style strands={4}{red}]
		a_3 a_1;
	\end{tikzpicture}
	
	$\sigma_3\sigma_1$
	\end{minipage}
	\vspace{0.2cm}
}

\uncover<3->{
	$\Leftrightarrow$
	
	\vspace{0.2cm}
	\begin{minipage}{0.15\linewidth}
  	\centering
	\begin{tikzpicture}[scale=0.5]
		\braid [number of strands=4,line width=2pt,style strands={1}{green},style strands={2}{yellow},style strands={3}{blue},style strands={4}{red}]
		a_1 a_3;
	\end{tikzpicture}
	
	$\sigma_1\sigma_3$
	\end{minipage}
}
\uncover<4->{
    *
	\begin{minipage}{0.15\linewidth}
		\centering
	\begin{tikzpicture}[scale=0.5]
		\braid [number of strands=4,line width=2pt,style strands={1}{green},style strands={2}{yellow},style strands={3}{blue},style strands={4}{red}]
		a_3^{-1} a_1^{-1};
	\end{tikzpicture}
	
	$\sigma_3^{-1}\sigma_1^{-1}$
	\end{minipage}
}	
\uncover<5->{
	=
	\begin{minipage}{0.15\linewidth}
		\centering
	\begin{tikzpicture}[scale=0.5]
		\braid [number of strands=4,line width=2pt,style strands={1}{green},style strands={2}{yellow},style strands={3}{blue},style strands={4}{red}]
		a_1 a_3 a_3^{-1} a_1^{-1};
	\end{tikzpicture}
	
	$\sigma_1\sigma_3\sigma_3^{-1}\sigma_1^{-1}$
	\end{minipage}
}
\uncover<6->{	
	=
	\begin{minipage}{0.15\linewidth}
		\centering
	\begin{tikzpicture}[scale=0.5]
		\braid [number of strands=4,line width=2pt,style strands={1}{green},style strands={2}{yellow},style strands={3}{blue},style strands={4}{red},height=4cm]
		1;
		
	\end{tikzpicture}
	
	$id$
	\end{minipage}
}
}

\frame{
\frametitle{Pure braids}
\uncover<1->{
%\begin{block}{Definition}
  \centering 
  \begin{minipage}{0.25\linewidth}
  	\centering
	\begin{tikzpicture}[scale=0.5]
		\braid [
			line width=2pt,
 			style strands={1}{green},
 			style strands={2}{yellow},
 			style strands={3}{blue},
 			style strands={4}{red}]
 			a_2 a_1^{-1} a_3 a_1^{-1} a_3 a_2 ;
	\end{tikzpicture}
	
	Pure braid
	}
\uncover<2->{
	\end{minipage}
	\hspace{0.5cm}
	\begin{minipage}{0.25\linewidth}
		\centering
	\begin{tikzpicture}[scale=0.5]
		\braid [
			line width=2pt,
 			style strands={1}{green},
 			style strands={2}{yellow},
 			style strands={3}{blue},
 			style strands={4}{red}]
 			a_1 a_2^{-1} a_1 a_3 a_2^{-1} a_3^{-1};
	\end{tikzpicture}
	
	Not pure
	\end{minipage}
}

\uncover<3->{
\begin{block}{Theorem}
The pure braid group $P_n$ is biorderable for all $n\geq 1$.
\end{block}
}
}

\frame{
\frametitle{Mapping class groups}
\uncover<1->{
\begin{block}{Definition}
Let $\mathcal{S}$ an oriented compact surface, possibly with boundary, and $\mathcal{P}$ be a finite set of distinguished interior points of $\mathcal{S}$.\\
}

\uncover<2->{
The \alert{mapping clas group} $\mathcal{MCG(S,P)}$ of the surface $\mathcal{S}$ relative to $\mathcal{P}$ is the group of all isotopy classes of orientation-preserving homeomorphisms $\psi:\mathcal{S}\rightarrow\mathcal{S}$ satisfying $\psi_{|\partial S}=id$ and $\psi(\mathcal{P})=\mathcal{P}$.
\end{block}
}

\uncover<3->{
\begin{block}{Proposition}
	There is an isomorphism $B_n\cong \mathcal{MCG}(D_n)$, where $D_n$ is the disk $D^2$ with $n$ regularly spaced points.
\end{block}
}
%\textcolor{red}{Dibujo?}
}

\frame{
\frametitle{Order in $B_n$}
$\sigma$-positive

subword property

\begin{block}{Proposition}

\end{block}

}

\frame{
\begin{tikzpicture}
\braid [
 style strands={1}{red},
 style strands={2}{blue},
 style strands={3}{green}]
 a_2 a_1^{-1} a_3 a_1^{-1} a_3 a_2 ;
\end{tikzpicture}
}

\frame{
\begin{tikzpicture}
\braid[color=red] | a_1 a_2;
\end{tikzpicture}
\begin{tikzpicture}
\braid a_1 | a_2;
\end{tikzpicture}
}

\frame{
\begin{tikzpicture}
\braid[
 style all floors={fill=yellow},
 style floors={1}{dashed,fill=yellow!50!green},
 floor command={%
 \fill (\floorsx,\floorsy) rectangle (\floorex,\floorey);
 \draw (\floorsx,\floorsy) -- (\floorex,\floorsy);
 },
 line width=2pt,
 style strands={1}{red},
 style strands={2}{blue},
 style strands={3}{green}
] (braid) at (2,0) | s_1-s_3-s_5 | s_2^{-1}-s_4| s_1-s_4 s_2^{-1} s_1-s_3 s_2^{-1}-s_4^{-1};
\fill[yellow] (2,0) circle (4pt);
\fill[purple] (braid) circle (4pt);
\node[at=(braid-3-s),pin=north west:strand 3] {};
\node[at=(braid-3-e),pin=south west:strand 3] {};
\node[at=(braid-rev-3-s),pin=north east:strand 3 (from bottom)] {};
\node[at=(braid-rev-3-e),pin=south east:strand 3 (from bottom)] {};
\end{tikzpicture}
}



\frame{
\begin{center}
\Huge Thanks.
\end{center}
}






\end{document}
